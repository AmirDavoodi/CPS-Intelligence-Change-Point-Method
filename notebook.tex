
% Default to the notebook output style

    


% Inherit from the specified cell style.




    
\documentclass[11pt]{article}

    
    
    \usepackage[T1]{fontenc}
    % Nicer default font (+ math font) than Computer Modern for most use cases
    \usepackage{mathpazo}

    % Basic figure setup, for now with no caption control since it's done
    % automatically by Pandoc (which extracts ![](path) syntax from Markdown).
    \usepackage{graphicx}
    % We will generate all images so they have a width \maxwidth. This means
    % that they will get their normal width if they fit onto the page, but
    % are scaled down if they would overflow the margins.
    \makeatletter
    \def\maxwidth{\ifdim\Gin@nat@width>\linewidth\linewidth
    \else\Gin@nat@width\fi}
    \makeatother
    \let\Oldincludegraphics\includegraphics
    % Set max figure width to be 80% of text width, for now hardcoded.
    \renewcommand{\includegraphics}[1]{\Oldincludegraphics[width=.8\maxwidth]{#1}}
    % Ensure that by default, figures have no caption (until we provide a
    % proper Figure object with a Caption API and a way to capture that
    % in the conversion process - todo).
    \usepackage{caption}
    \DeclareCaptionLabelFormat{nolabel}{}
    \captionsetup{labelformat=nolabel}

    \usepackage{adjustbox} % Used to constrain images to a maximum size 
    \usepackage{xcolor} % Allow colors to be defined
    \usepackage{enumerate} % Needed for markdown enumerations to work
    \usepackage{geometry} % Used to adjust the document margins
    \usepackage{amsmath} % Equations
    \usepackage{amssymb} % Equations
    \usepackage{textcomp} % defines textquotesingle
    % Hack from http://tex.stackexchange.com/a/47451/13684:
    \AtBeginDocument{%
        \def\PYZsq{\textquotesingle}% Upright quotes in Pygmentized code
    }
    \usepackage{upquote} % Upright quotes for verbatim code
    \usepackage{eurosym} % defines \euro
    \usepackage[mathletters]{ucs} % Extended unicode (utf-8) support
    \usepackage[utf8x]{inputenc} % Allow utf-8 characters in the tex document
    \usepackage{fancyvrb} % verbatim replacement that allows latex
    \usepackage{grffile} % extends the file name processing of package graphics 
                         % to support a larger range 
    % The hyperref package gives us a pdf with properly built
    % internal navigation ('pdf bookmarks' for the table of contents,
    % internal cross-reference links, web links for URLs, etc.)
    \usepackage{hyperref}
    \usepackage{longtable} % longtable support required by pandoc >1.10
    \usepackage{booktabs}  % table support for pandoc > 1.12.2
    \usepackage[inline]{enumitem} % IRkernel/repr support (it uses the enumerate* environment)
    \usepackage[normalem]{ulem} % ulem is needed to support strikethroughs (\sout)
                                % normalem makes italics be italics, not underlines
    

    
    
    % Colors for the hyperref package
    \definecolor{urlcolor}{rgb}{0,.145,.698}
    \definecolor{linkcolor}{rgb}{.71,0.21,0.01}
    \definecolor{citecolor}{rgb}{.12,.54,.11}

    % ANSI colors
    \definecolor{ansi-black}{HTML}{3E424D}
    \definecolor{ansi-black-intense}{HTML}{282C36}
    \definecolor{ansi-red}{HTML}{E75C58}
    \definecolor{ansi-red-intense}{HTML}{B22B31}
    \definecolor{ansi-green}{HTML}{00A250}
    \definecolor{ansi-green-intense}{HTML}{007427}
    \definecolor{ansi-yellow}{HTML}{DDB62B}
    \definecolor{ansi-yellow-intense}{HTML}{B27D12}
    \definecolor{ansi-blue}{HTML}{208FFB}
    \definecolor{ansi-blue-intense}{HTML}{0065CA}
    \definecolor{ansi-magenta}{HTML}{D160C4}
    \definecolor{ansi-magenta-intense}{HTML}{A03196}
    \definecolor{ansi-cyan}{HTML}{60C6C8}
    \definecolor{ansi-cyan-intense}{HTML}{258F8F}
    \definecolor{ansi-white}{HTML}{C5C1B4}
    \definecolor{ansi-white-intense}{HTML}{A1A6B2}

    % commands and environments needed by pandoc snippets
    % extracted from the output of `pandoc -s`
    \providecommand{\tightlist}{%
      \setlength{\itemsep}{0pt}\setlength{\parskip}{0pt}}
    \DefineVerbatimEnvironment{Highlighting}{Verbatim}{commandchars=\\\{\}}
    % Add ',fontsize=\small' for more characters per line
    \newenvironment{Shaded}{}{}
    \newcommand{\KeywordTok}[1]{\textcolor[rgb]{0.00,0.44,0.13}{\textbf{{#1}}}}
    \newcommand{\DataTypeTok}[1]{\textcolor[rgb]{0.56,0.13,0.00}{{#1}}}
    \newcommand{\DecValTok}[1]{\textcolor[rgb]{0.25,0.63,0.44}{{#1}}}
    \newcommand{\BaseNTok}[1]{\textcolor[rgb]{0.25,0.63,0.44}{{#1}}}
    \newcommand{\FloatTok}[1]{\textcolor[rgb]{0.25,0.63,0.44}{{#1}}}
    \newcommand{\CharTok}[1]{\textcolor[rgb]{0.25,0.44,0.63}{{#1}}}
    \newcommand{\StringTok}[1]{\textcolor[rgb]{0.25,0.44,0.63}{{#1}}}
    \newcommand{\CommentTok}[1]{\textcolor[rgb]{0.38,0.63,0.69}{\textit{{#1}}}}
    \newcommand{\OtherTok}[1]{\textcolor[rgb]{0.00,0.44,0.13}{{#1}}}
    \newcommand{\AlertTok}[1]{\textcolor[rgb]{1.00,0.00,0.00}{\textbf{{#1}}}}
    \newcommand{\FunctionTok}[1]{\textcolor[rgb]{0.02,0.16,0.49}{{#1}}}
    \newcommand{\RegionMarkerTok}[1]{{#1}}
    \newcommand{\ErrorTok}[1]{\textcolor[rgb]{1.00,0.00,0.00}{\textbf{{#1}}}}
    \newcommand{\NormalTok}[1]{{#1}}
    
    % Additional commands for more recent versions of Pandoc
    \newcommand{\ConstantTok}[1]{\textcolor[rgb]{0.53,0.00,0.00}{{#1}}}
    \newcommand{\SpecialCharTok}[1]{\textcolor[rgb]{0.25,0.44,0.63}{{#1}}}
    \newcommand{\VerbatimStringTok}[1]{\textcolor[rgb]{0.25,0.44,0.63}{{#1}}}
    \newcommand{\SpecialStringTok}[1]{\textcolor[rgb]{0.73,0.40,0.53}{{#1}}}
    \newcommand{\ImportTok}[1]{{#1}}
    \newcommand{\DocumentationTok}[1]{\textcolor[rgb]{0.73,0.13,0.13}{\textit{{#1}}}}
    \newcommand{\AnnotationTok}[1]{\textcolor[rgb]{0.38,0.63,0.69}{\textbf{\textit{{#1}}}}}
    \newcommand{\CommentVarTok}[1]{\textcolor[rgb]{0.38,0.63,0.69}{\textbf{\textit{{#1}}}}}
    \newcommand{\VariableTok}[1]{\textcolor[rgb]{0.10,0.09,0.49}{{#1}}}
    \newcommand{\ControlFlowTok}[1]{\textcolor[rgb]{0.00,0.44,0.13}{\textbf{{#1}}}}
    \newcommand{\OperatorTok}[1]{\textcolor[rgb]{0.40,0.40,0.40}{{#1}}}
    \newcommand{\BuiltInTok}[1]{{#1}}
    \newcommand{\ExtensionTok}[1]{{#1}}
    \newcommand{\PreprocessorTok}[1]{\textcolor[rgb]{0.74,0.48,0.00}{{#1}}}
    \newcommand{\AttributeTok}[1]{\textcolor[rgb]{0.49,0.56,0.16}{{#1}}}
    \newcommand{\InformationTok}[1]{\textcolor[rgb]{0.38,0.63,0.69}{\textbf{\textit{{#1}}}}}
    \newcommand{\WarningTok}[1]{\textcolor[rgb]{0.38,0.63,0.69}{\textbf{\textit{{#1}}}}}
    
    
    % Define a nice break command that doesn't care if a line doesn't already
    % exist.
    \def\br{\hspace*{\fill} \\* }
    % Math Jax compatability definitions
    \def\gt{>}
    \def\lt{<}
    % Document parameters
    \title{CPM}
    
    
    

    % Pygments definitions
    
\makeatletter
\def\PY@reset{\let\PY@it=\relax \let\PY@bf=\relax%
    \let\PY@ul=\relax \let\PY@tc=\relax%
    \let\PY@bc=\relax \let\PY@ff=\relax}
\def\PY@tok#1{\csname PY@tok@#1\endcsname}
\def\PY@toks#1+{\ifx\relax#1\empty\else%
    \PY@tok{#1}\expandafter\PY@toks\fi}
\def\PY@do#1{\PY@bc{\PY@tc{\PY@ul{%
    \PY@it{\PY@bf{\PY@ff{#1}}}}}}}
\def\PY#1#2{\PY@reset\PY@toks#1+\relax+\PY@do{#2}}

\expandafter\def\csname PY@tok@w\endcsname{\def\PY@tc##1{\textcolor[rgb]{0.73,0.73,0.73}{##1}}}
\expandafter\def\csname PY@tok@c\endcsname{\let\PY@it=\textit\def\PY@tc##1{\textcolor[rgb]{0.25,0.50,0.50}{##1}}}
\expandafter\def\csname PY@tok@cp\endcsname{\def\PY@tc##1{\textcolor[rgb]{0.74,0.48,0.00}{##1}}}
\expandafter\def\csname PY@tok@k\endcsname{\let\PY@bf=\textbf\def\PY@tc##1{\textcolor[rgb]{0.00,0.50,0.00}{##1}}}
\expandafter\def\csname PY@tok@kp\endcsname{\def\PY@tc##1{\textcolor[rgb]{0.00,0.50,0.00}{##1}}}
\expandafter\def\csname PY@tok@kt\endcsname{\def\PY@tc##1{\textcolor[rgb]{0.69,0.00,0.25}{##1}}}
\expandafter\def\csname PY@tok@o\endcsname{\def\PY@tc##1{\textcolor[rgb]{0.40,0.40,0.40}{##1}}}
\expandafter\def\csname PY@tok@ow\endcsname{\let\PY@bf=\textbf\def\PY@tc##1{\textcolor[rgb]{0.67,0.13,1.00}{##1}}}
\expandafter\def\csname PY@tok@nb\endcsname{\def\PY@tc##1{\textcolor[rgb]{0.00,0.50,0.00}{##1}}}
\expandafter\def\csname PY@tok@nf\endcsname{\def\PY@tc##1{\textcolor[rgb]{0.00,0.00,1.00}{##1}}}
\expandafter\def\csname PY@tok@nc\endcsname{\let\PY@bf=\textbf\def\PY@tc##1{\textcolor[rgb]{0.00,0.00,1.00}{##1}}}
\expandafter\def\csname PY@tok@nn\endcsname{\let\PY@bf=\textbf\def\PY@tc##1{\textcolor[rgb]{0.00,0.00,1.00}{##1}}}
\expandafter\def\csname PY@tok@ne\endcsname{\let\PY@bf=\textbf\def\PY@tc##1{\textcolor[rgb]{0.82,0.25,0.23}{##1}}}
\expandafter\def\csname PY@tok@nv\endcsname{\def\PY@tc##1{\textcolor[rgb]{0.10,0.09,0.49}{##1}}}
\expandafter\def\csname PY@tok@no\endcsname{\def\PY@tc##1{\textcolor[rgb]{0.53,0.00,0.00}{##1}}}
\expandafter\def\csname PY@tok@nl\endcsname{\def\PY@tc##1{\textcolor[rgb]{0.63,0.63,0.00}{##1}}}
\expandafter\def\csname PY@tok@ni\endcsname{\let\PY@bf=\textbf\def\PY@tc##1{\textcolor[rgb]{0.60,0.60,0.60}{##1}}}
\expandafter\def\csname PY@tok@na\endcsname{\def\PY@tc##1{\textcolor[rgb]{0.49,0.56,0.16}{##1}}}
\expandafter\def\csname PY@tok@nt\endcsname{\let\PY@bf=\textbf\def\PY@tc##1{\textcolor[rgb]{0.00,0.50,0.00}{##1}}}
\expandafter\def\csname PY@tok@nd\endcsname{\def\PY@tc##1{\textcolor[rgb]{0.67,0.13,1.00}{##1}}}
\expandafter\def\csname PY@tok@s\endcsname{\def\PY@tc##1{\textcolor[rgb]{0.73,0.13,0.13}{##1}}}
\expandafter\def\csname PY@tok@sd\endcsname{\let\PY@it=\textit\def\PY@tc##1{\textcolor[rgb]{0.73,0.13,0.13}{##1}}}
\expandafter\def\csname PY@tok@si\endcsname{\let\PY@bf=\textbf\def\PY@tc##1{\textcolor[rgb]{0.73,0.40,0.53}{##1}}}
\expandafter\def\csname PY@tok@se\endcsname{\let\PY@bf=\textbf\def\PY@tc##1{\textcolor[rgb]{0.73,0.40,0.13}{##1}}}
\expandafter\def\csname PY@tok@sr\endcsname{\def\PY@tc##1{\textcolor[rgb]{0.73,0.40,0.53}{##1}}}
\expandafter\def\csname PY@tok@ss\endcsname{\def\PY@tc##1{\textcolor[rgb]{0.10,0.09,0.49}{##1}}}
\expandafter\def\csname PY@tok@sx\endcsname{\def\PY@tc##1{\textcolor[rgb]{0.00,0.50,0.00}{##1}}}
\expandafter\def\csname PY@tok@m\endcsname{\def\PY@tc##1{\textcolor[rgb]{0.40,0.40,0.40}{##1}}}
\expandafter\def\csname PY@tok@gh\endcsname{\let\PY@bf=\textbf\def\PY@tc##1{\textcolor[rgb]{0.00,0.00,0.50}{##1}}}
\expandafter\def\csname PY@tok@gu\endcsname{\let\PY@bf=\textbf\def\PY@tc##1{\textcolor[rgb]{0.50,0.00,0.50}{##1}}}
\expandafter\def\csname PY@tok@gd\endcsname{\def\PY@tc##1{\textcolor[rgb]{0.63,0.00,0.00}{##1}}}
\expandafter\def\csname PY@tok@gi\endcsname{\def\PY@tc##1{\textcolor[rgb]{0.00,0.63,0.00}{##1}}}
\expandafter\def\csname PY@tok@gr\endcsname{\def\PY@tc##1{\textcolor[rgb]{1.00,0.00,0.00}{##1}}}
\expandafter\def\csname PY@tok@ge\endcsname{\let\PY@it=\textit}
\expandafter\def\csname PY@tok@gs\endcsname{\let\PY@bf=\textbf}
\expandafter\def\csname PY@tok@gp\endcsname{\let\PY@bf=\textbf\def\PY@tc##1{\textcolor[rgb]{0.00,0.00,0.50}{##1}}}
\expandafter\def\csname PY@tok@go\endcsname{\def\PY@tc##1{\textcolor[rgb]{0.53,0.53,0.53}{##1}}}
\expandafter\def\csname PY@tok@gt\endcsname{\def\PY@tc##1{\textcolor[rgb]{0.00,0.27,0.87}{##1}}}
\expandafter\def\csname PY@tok@err\endcsname{\def\PY@bc##1{\setlength{\fboxsep}{0pt}\fcolorbox[rgb]{1.00,0.00,0.00}{1,1,1}{\strut ##1}}}
\expandafter\def\csname PY@tok@kc\endcsname{\let\PY@bf=\textbf\def\PY@tc##1{\textcolor[rgb]{0.00,0.50,0.00}{##1}}}
\expandafter\def\csname PY@tok@kd\endcsname{\let\PY@bf=\textbf\def\PY@tc##1{\textcolor[rgb]{0.00,0.50,0.00}{##1}}}
\expandafter\def\csname PY@tok@kn\endcsname{\let\PY@bf=\textbf\def\PY@tc##1{\textcolor[rgb]{0.00,0.50,0.00}{##1}}}
\expandafter\def\csname PY@tok@kr\endcsname{\let\PY@bf=\textbf\def\PY@tc##1{\textcolor[rgb]{0.00,0.50,0.00}{##1}}}
\expandafter\def\csname PY@tok@bp\endcsname{\def\PY@tc##1{\textcolor[rgb]{0.00,0.50,0.00}{##1}}}
\expandafter\def\csname PY@tok@fm\endcsname{\def\PY@tc##1{\textcolor[rgb]{0.00,0.00,1.00}{##1}}}
\expandafter\def\csname PY@tok@vc\endcsname{\def\PY@tc##1{\textcolor[rgb]{0.10,0.09,0.49}{##1}}}
\expandafter\def\csname PY@tok@vg\endcsname{\def\PY@tc##1{\textcolor[rgb]{0.10,0.09,0.49}{##1}}}
\expandafter\def\csname PY@tok@vi\endcsname{\def\PY@tc##1{\textcolor[rgb]{0.10,0.09,0.49}{##1}}}
\expandafter\def\csname PY@tok@vm\endcsname{\def\PY@tc##1{\textcolor[rgb]{0.10,0.09,0.49}{##1}}}
\expandafter\def\csname PY@tok@sa\endcsname{\def\PY@tc##1{\textcolor[rgb]{0.73,0.13,0.13}{##1}}}
\expandafter\def\csname PY@tok@sb\endcsname{\def\PY@tc##1{\textcolor[rgb]{0.73,0.13,0.13}{##1}}}
\expandafter\def\csname PY@tok@sc\endcsname{\def\PY@tc##1{\textcolor[rgb]{0.73,0.13,0.13}{##1}}}
\expandafter\def\csname PY@tok@dl\endcsname{\def\PY@tc##1{\textcolor[rgb]{0.73,0.13,0.13}{##1}}}
\expandafter\def\csname PY@tok@s2\endcsname{\def\PY@tc##1{\textcolor[rgb]{0.73,0.13,0.13}{##1}}}
\expandafter\def\csname PY@tok@sh\endcsname{\def\PY@tc##1{\textcolor[rgb]{0.73,0.13,0.13}{##1}}}
\expandafter\def\csname PY@tok@s1\endcsname{\def\PY@tc##1{\textcolor[rgb]{0.73,0.13,0.13}{##1}}}
\expandafter\def\csname PY@tok@mb\endcsname{\def\PY@tc##1{\textcolor[rgb]{0.40,0.40,0.40}{##1}}}
\expandafter\def\csname PY@tok@mf\endcsname{\def\PY@tc##1{\textcolor[rgb]{0.40,0.40,0.40}{##1}}}
\expandafter\def\csname PY@tok@mh\endcsname{\def\PY@tc##1{\textcolor[rgb]{0.40,0.40,0.40}{##1}}}
\expandafter\def\csname PY@tok@mi\endcsname{\def\PY@tc##1{\textcolor[rgb]{0.40,0.40,0.40}{##1}}}
\expandafter\def\csname PY@tok@il\endcsname{\def\PY@tc##1{\textcolor[rgb]{0.40,0.40,0.40}{##1}}}
\expandafter\def\csname PY@tok@mo\endcsname{\def\PY@tc##1{\textcolor[rgb]{0.40,0.40,0.40}{##1}}}
\expandafter\def\csname PY@tok@ch\endcsname{\let\PY@it=\textit\def\PY@tc##1{\textcolor[rgb]{0.25,0.50,0.50}{##1}}}
\expandafter\def\csname PY@tok@cm\endcsname{\let\PY@it=\textit\def\PY@tc##1{\textcolor[rgb]{0.25,0.50,0.50}{##1}}}
\expandafter\def\csname PY@tok@cpf\endcsname{\let\PY@it=\textit\def\PY@tc##1{\textcolor[rgb]{0.25,0.50,0.50}{##1}}}
\expandafter\def\csname PY@tok@c1\endcsname{\let\PY@it=\textit\def\PY@tc##1{\textcolor[rgb]{0.25,0.50,0.50}{##1}}}
\expandafter\def\csname PY@tok@cs\endcsname{\let\PY@it=\textit\def\PY@tc##1{\textcolor[rgb]{0.25,0.50,0.50}{##1}}}

\def\PYZbs{\char`\\}
\def\PYZus{\char`\_}
\def\PYZob{\char`\{}
\def\PYZcb{\char`\}}
\def\PYZca{\char`\^}
\def\PYZam{\char`\&}
\def\PYZlt{\char`\<}
\def\PYZgt{\char`\>}
\def\PYZsh{\char`\#}
\def\PYZpc{\char`\%}
\def\PYZdl{\char`\$}
\def\PYZhy{\char`\-}
\def\PYZsq{\char`\'}
\def\PYZdq{\char`\"}
\def\PYZti{\char`\~}
% for compatibility with earlier versions
\def\PYZat{@}
\def\PYZlb{[}
\def\PYZrb{]}
\makeatother


    % Exact colors from NB
    \definecolor{incolor}{rgb}{0.0, 0.0, 0.5}
    \definecolor{outcolor}{rgb}{0.545, 0.0, 0.0}



    
    % Prevent overflowing lines due to hard-to-break entities
    \sloppy 
    % Setup hyperref package
    \hypersetup{
      breaklinks=true,  % so long urls are correctly broken across lines
      colorlinks=true,
      urlcolor=urlcolor,
      linkcolor=linkcolor,
      citecolor=citecolor,
      }
    % Slightly bigger margins than the latex defaults
    
    \geometry{verbose,tmargin=1in,bmargin=1in,lmargin=1in,rmargin=1in}
    
    

    \begin{document}
    
    
    \maketitle
    
    

    
    \hypertarget{change-point-method}{%
\section{Change Point Method}\label{change-point-method}}

Consider a sequence \(x_1\), \(x_2\), \(\ldots\), \(x_T\) as
observations of independent Gaussian random variables \(X_1\), \(X_2\),
\ldots{}, \(X_T\). The goal is to identify whether all \(x_i\) ,
\(i=1,\ldots,T\) have been generated by the same Gaussian distribution
\(N(\eta_0,\sigma^2)\), or a change in stationarity occurred at certain
point \(t^\ast\) so that \(x_i\), \(i < t^\ast\) is drawn from
\(N(\eta_0,\sigma^2)\) and \(x_i\), \(i≥t^\ast\) from
\(N(\eta_1,\sigma^2)\). In case a change is present, we also want to
estimate the change time \(t^\ast\).

    \hypertarget{implement-a-cpm}{%
\subsection{Implement a CPM}\label{implement-a-cpm}}

To answer the above questions, we implement the Change Point Method
(CPM) described below.

Define a function my\_cpm(x, alpha) that applies the CPM described below
to sequence \(x=[x_1, x_2, \cdots, x_T]\) of length \(T\), and returns
None if no significant change is detected, otherwise it returns the
estimated time \(t_e\).

Scheme of a CPM:

For each \(t\) from \(1\) to \(T-1\) apply a two-sample t-test to
sub-sequences \([x_1, \cdots, x_t]\) and \([x_t+1, \cdots, x_T]\) and
store the obtained statistic \(S(t)\) and p-value \(p(t)\). (Hint: you
can use method scipy.stats.ttest\_1samp)

Select time \(t_e=argmax\) \(S(t)\) where statistic is maximal and
consider it as candidate to be the estimated time of the change.

compare \(p(t_e)\) with significance level \(\alpha\) to decide if there
is statistical evidence of a change at time \(t_e\).

Return \(t_e\) if a significant change is detected, None otherwise.

    \begin{Verbatim}[commandchars=\\\{\}]
{\color{incolor}In [{\color{incolor}174}]:} \PY{o}{\PYZpc{}}\PY{k}{matplotlib} inline
          \PY{k+kn}{import} \PY{n+nn}{numpy} \PY{k}{as} \PY{n+nn}{np}
          \PY{k+kn}{import} \PY{n+nn}{matplotlib}\PY{n+nn}{.}\PY{n+nn}{pyplot} \PY{k}{as} \PY{n+nn}{plt}
          \PY{k+kn}{from} \PY{n+nn}{matplotlib}\PY{n+nn}{.}\PY{n+nn}{gridspec} \PY{k}{import} \PY{n}{GridSpec}
          
          \PY{k+kn}{import} \PY{n+nn}{numpy} \PY{k}{as} \PY{n+nn}{np}
          \PY{k+kn}{from} \PY{n+nn}{scipy} \PY{k}{import} \PY{n}{stats}
          
          
          \PY{l+s+sd}{\PYZsq{}\PYZsq{}\PYZsq{} }
          \PY{l+s+sd}{parameters:}
          \PY{l+s+sd}{    x = an array of values for the random variable X from x\PYZus{}1 to x\PYZus{}T}
          \PY{l+s+sd}{    alpha = significance level alpha which in default I consider its value as 0.05}
          \PY{l+s+sd}{            Just for being match in notation I assumed (alpha = 1 \PYZhy{} p\PYZhy{}value) so}
          \PY{l+s+sd}{            if alpha = 0.05 it means p\PYZhy{}value = 0.95}
          \PY{l+s+sd}{returns None if no significant change is detected, otherwise it returns the estimated time t\PYZus{}e}
          \PY{l+s+sd}{\PYZsq{}\PYZsq{}\PYZsq{}}
          \PY{k}{def} \PY{n+nf}{my\PYZus{}cpm}\PY{p}{(}\PY{n}{x}\PY{p}{,} \PY{n}{alpha}\PY{o}{=}\PY{l+m+mf}{0.05}\PY{p}{)}\PY{p}{:}
              \PY{n}{T} \PY{o}{=} \PY{n+nb}{len}\PY{p}{(}\PY{n}{x}\PY{p}{)}
              \PY{c+c1}{\PYZsh{} place holder for statistic S(t) and p\PYZhy{}value p(t)}
              \PY{n}{statistic} \PY{o}{=} \PY{p}{[}\PY{p}{]}
              \PY{n}{pvalue} \PY{o}{=} \PY{p}{[}\PY{p}{]}
              
              \PY{k}{for} \PY{n}{t} \PY{o+ow}{in} \PY{n+nb}{range}\PY{p}{(}\PY{l+m+mi}{2}\PY{p}{,}\PY{n}{T}\PY{o}{\PYZhy{}}\PY{l+m+mi}{1}\PY{p}{)}\PY{p}{:}
                  \PY{l+s+sd}{\PYZsq{}\PYZsq{}\PYZsq{} }
          \PY{l+s+sd}{            This is a two\PYZhy{}sided test for the null hypothesis that 2 independent}
          \PY{l+s+sd}{            samples have identical average (expected) values.}
          \PY{l+s+sd}{            This test assumes that the populations have identical variances by default.}
          \PY{l+s+sd}{        \PYZsq{}\PYZsq{}\PYZsq{}} 
                  \PY{c+c1}{\PYZsh{} }
                  \PY{c+c1}{\PYZsh{} }
                  \PY{n}{s}\PY{p}{,} \PY{n}{p} \PY{o}{=} \PY{n}{stats}\PY{o}{.}\PY{n}{ttest\PYZus{}ind}\PY{p}{(}\PY{n}{x}\PY{p}{[}\PY{l+m+mi}{0}\PY{p}{:}\PY{n}{t}\PY{p}{]}\PY{p}{,}\PY{n}{x}\PY{p}{[}\PY{n}{t}\PY{p}{:}\PY{p}{]}\PY{p}{)}
                  \PY{n}{statistic}\PY{o}{.}\PY{n}{append}\PY{p}{(}\PY{n}{s}\PY{p}{)}
                  \PY{n}{pvalue}\PY{o}{.}\PY{n}{append}\PY{p}{(}\PY{n}{p}\PY{p}{)}
              
              \PY{n}{t\PYZus{}e} \PY{o}{=} \PY{n}{np}\PY{o}{.}\PY{n}{argmax}\PY{p}{(}\PY{n}{np}\PY{o}{.}\PY{n}{absolute}\PY{p}{(}\PY{n}{statistic}\PY{p}{)}\PY{p}{)}
              
              \PY{k}{if} \PY{n}{pvalue}\PY{p}{[}\PY{n}{t\PYZus{}e}\PY{p}{]} \PY{o}{\PYZlt{}} \PY{n}{alpha}\PY{p}{:}
                  \PY{c+c1}{\PYZsh{} reject null hypothesis (change happening)}
                  \PY{k}{return} \PY{n}{t\PYZus{}e}\PY{o}{+}\PY{l+m+mi}{2}
              \PY{k}{else}\PY{p}{:}
                  \PY{c+c1}{\PYZsh{} no statistical evidence to reject null hypothesis (no change happening)}
                  \PY{k}{return} \PY{k+kc}{None}
\end{Verbatim}


    \hypertarget{some-explanation-about-the-python-implementation}{%
\subsubsection{Some explanation about the python
implementation}\label{some-explanation-about-the-python-implementation}}

For better implementation, I consider the changing point time \(t\) from
\(2\) to \(T-2\) because it is practically impossible to calculate the
statistic test \(S(t)\) and p-value \(p(t)\) test between one array and
one single value. Also in concept I think it would be better to do not
consider first element of our observation sequence as a changing point.

For the same reason, I sum the return value with number \(2\) because I
was using a local array variable inside this function which the first
element of my array is corresponding to the third element of the
observation sequence.

The test statistic tells us how much the two sample mean deviates from
each other assuming for the null hypothesis that 2 independent samples
have identical average (expected) values. Since the sample could differ
in either the positive or negative direction, for computing the
\(argmax\) of \(S(t)\) I consider the absolute value of this statistic
test \(S(t)\). In other worlds, the derivation can be in both positive
an negative direction and I want to find the time \(t\) which gives me
the two arrays which deviate most from each others.

The test yields a p-value which means there is a \(p(t)\) chance I'd see
sample data this far apart if the two groups tested are actually
identical. As I was using a \(95\%\) confidence level as default value I
would fail to reject the null hypothesis if the p-value is greater or
equal than the corresponding significance level of \(5\%\).

    \hypertarget{test-the-cpm-on-a-sequence}{%
\subsection{Test the CPM on a
sequence}\label{test-the-cpm-on-a-sequence}}

Generate a sequence \(x_1, x_2, ..., x_T\) by sampling the first \(200\)
real numbers from \(N(0,2)\), then other \(100\) from \(N(1,2)\); this
means \(T=300\) and \(t^*=200\).

Apply my\_cpm(x, alpha) to the generated sequence with signifcance level
\(\alpha=0.05\).

Plot statistic \(S(t)\) and p-value \(p(t)\) as function of time.

Does the CPM detected a change? What is the associated confidence level?

    \begin{Verbatim}[commandchars=\\\{\}]
{\color{incolor}In [{\color{incolor}175}]:} \PY{c+c1}{\PYZsh{} fix seed to get the same result}
          \PY{n}{np}\PY{o}{.}\PY{n}{random}\PY{o}{.}\PY{n}{seed}\PY{p}{(}\PY{l+m+mi}{12345678}\PY{p}{)}
          
          \PY{c+c1}{\PYZsh{} for generating sample I am using \PYZdq{}stats.norm.rvs\PYZdq{} function which have the following parameters}
          \PY{c+c1}{\PYZsh{} \PYZhy{}\PYZhy{}\PYZhy{}\PYZhy{}\PYZhy{}}
          \PY{c+c1}{\PYZsh{} loc = The location (loc) keyword specifies the mean of the sample}
          \PY{c+c1}{\PYZsh{} scale = The scale (scale) keyword specifies the standard deviation}
          \PY{c+c1}{\PYZsh{} size = the size of the sample or number of different random generated numbers}
          \PY{c+c1}{\PYZsh{} first200samples = stats.norm.rvs(loc=0,scale=2,size=200)}
          \PY{n}{mu}\PY{o}{=}\PY{l+m+mi}{0}
          \PY{n}{sigma}\PY{o}{=}\PY{l+m+mi}{2}
          \PY{n}{first200samples} \PY{o}{=} \PY{n}{np}\PY{o}{.}\PY{n}{random}\PY{o}{.}\PY{n}{normal}\PY{p}{(}\PY{n}{mu}\PY{p}{,} \PY{n}{sigma}\PY{p}{,} \PY{l+m+mi}{200}\PY{p}{)}
          \PY{c+c1}{\PYZsh{} print(\PYZdq{}first samples\PYZdq{})}
          \PY{c+c1}{\PYZsh{} print(first200samples)}
          \PY{c+c1}{\PYZsh{} second100samples = stats.norm.rvs(loc=1,scale=2,size=100)}
          \PY{n}{mu}\PY{o}{=}\PY{l+m+mi}{1}
          \PY{n}{sigma}\PY{o}{=}\PY{l+m+mi}{2}
          \PY{n}{second100samples} \PY{o}{=} \PY{n}{np}\PY{o}{.}\PY{n}{random}\PY{o}{.}\PY{n}{normal}\PY{p}{(}\PY{n}{mu}\PY{p}{,} \PY{n}{sigma}\PY{p}{,} \PY{l+m+mi}{100}\PY{p}{)}
          \PY{c+c1}{\PYZsh{} print(\PYZdq{}second samples\PYZdq{})}
          \PY{c+c1}{\PYZsh{} print(second100samples)}
          \PY{n}{t\PYZus{}star} \PY{o}{=} \PY{l+m+mi}{200}
          \PY{n}{samples} \PY{o}{=} \PY{n}{np}\PY{o}{.}\PY{n}{concatenate}\PY{p}{(}\PY{p}{(}\PY{n}{first200samples}\PY{p}{,}\PY{n}{second100samples}\PY{p}{)}\PY{p}{)}
          \PY{c+c1}{\PYZsh{} print(\PYZdq{}concatinated samples\PYZdq{})}
          \PY{c+c1}{\PYZsh{} print(samples)}
          
          \PY{n+nb}{print}\PY{p}{(}\PY{l+s+s2}{\PYZdq{}}\PY{l+s+s2}{The computed change point index (t\PYZus{}e) = }\PY{l+s+s2}{\PYZdq{}}\PY{p}{,}\PY{n}{my\PYZus{}cpm}\PY{p}{(}\PY{n}{samples}\PY{p}{,} \PY{n}{alpha}\PY{o}{=}\PY{l+m+mf}{0.05}\PY{p}{)}\PY{p}{)}
          
          
          \PY{l+s+sd}{\PYZsq{}\PYZsq{}\PYZsq{} Ploting statistic S(t) and p\PYZhy{}value p(t) as a function of time\PYZsq{}\PYZsq{}\PYZsq{}}
          \PY{k}{def} \PY{n+nf}{t\PYZus{}test\PYZus{}plots}\PY{p}{(}\PY{n}{x}\PY{p}{,} \PY{n}{alpha}\PY{o}{=}\PY{l+m+mf}{0.05}\PY{p}{)}\PY{p}{:}
              \PY{n}{T} \PY{o}{=} \PY{n+nb}{len}\PY{p}{(}\PY{n}{x}\PY{p}{)}
              \PY{c+c1}{\PYZsh{} place holder for statistic S(t) and p\PYZhy{}value p(t)}
              \PY{n}{statistic} \PY{o}{=} \PY{p}{[}\PY{p}{]}
              \PY{n}{pvalue} \PY{o}{=} \PY{p}{[}\PY{p}{]}
              
              \PY{k}{for} \PY{n}{t} \PY{o+ow}{in} \PY{n+nb}{range}\PY{p}{(}\PY{l+m+mi}{2}\PY{p}{,}\PY{n}{T}\PY{o}{\PYZhy{}}\PY{l+m+mi}{1}\PY{p}{)}\PY{p}{:}
                  \PY{n}{s}\PY{p}{,} \PY{n}{p} \PY{o}{=} \PY{n}{stats}\PY{o}{.}\PY{n}{ttest\PYZus{}ind}\PY{p}{(}\PY{n}{x}\PY{p}{[}\PY{l+m+mi}{0}\PY{p}{:}\PY{n}{t}\PY{p}{]}\PY{p}{,}\PY{n}{x}\PY{p}{[}\PY{n}{t}\PY{p}{:}\PY{p}{]}\PY{p}{)}
                  \PY{n}{statistic}\PY{o}{.}\PY{n}{append}\PY{p}{(}\PY{n}{s}\PY{p}{)}
                  \PY{n}{pvalue}\PY{o}{.}\PY{n}{append}\PY{p}{(}\PY{n}{p}\PY{p}{)}
                  
              \PY{n}{t\PYZus{}e} \PY{o}{=} \PY{n}{np}\PY{o}{.}\PY{n}{argmax}\PY{p}{(}\PY{n}{np}\PY{o}{.}\PY{n}{absolute}\PY{p}{(}\PY{n}{statistic}\PY{p}{)}\PY{p}{)}
              
              \PY{n}{time\PYZus{}array} \PY{o}{=} \PY{n+nb}{range}\PY{p}{(}\PY{l+m+mi}{2}\PY{p}{,}\PY{n}{T}\PY{o}{\PYZhy{}}\PY{l+m+mi}{1}\PY{p}{)}
              \PY{n}{fig}\PY{p}{,} \PY{p}{(}\PY{n}{plot1}\PY{p}{,} \PY{n}{plot2}\PY{p}{)} \PY{o}{=} \PY{n}{plt}\PY{o}{.}\PY{n}{subplots}\PY{p}{(}\PY{n}{nrows}\PY{o}{=}\PY{l+m+mi}{2}\PY{p}{,} \PY{n}{ncols}\PY{o}{=}\PY{l+m+mi}{1}\PY{p}{,} \PY{n}{figsize}\PY{o}{=}\PY{p}{(}\PY{l+m+mi}{16}\PY{p}{,}\PY{l+m+mi}{5}\PY{p}{)}\PY{p}{)}
              \PY{n}{plot1}\PY{o}{.}\PY{n}{set\PYZus{}xlabel}\PY{p}{(}\PY{l+s+s1}{\PYZsq{}}\PY{l+s+s1}{time}\PY{l+s+s1}{\PYZsq{}}\PY{p}{)}
              \PY{n}{plot1}\PY{o}{.}\PY{n}{set\PYZus{}ylabel}\PY{p}{(}\PY{l+s+s1}{\PYZsq{}}\PY{l+s+s1}{Statistic S(t)}\PY{l+s+s1}{\PYZsq{}}\PY{p}{)}
              \PY{n}{plot1}\PY{o}{.}\PY{n}{plot}\PY{p}{(}\PY{n}{time\PYZus{}array}\PY{p}{,} \PY{n}{statistic}\PY{p}{,}\PY{l+s+s1}{\PYZsq{}}\PY{l+s+s1}{b}\PY{l+s+s1}{\PYZsq{}}\PY{p}{)}
              \PY{n}{plot2}\PY{o}{.}\PY{n}{set\PYZus{}xlabel}\PY{p}{(}\PY{l+s+s1}{\PYZsq{}}\PY{l+s+s1}{time}\PY{l+s+s1}{\PYZsq{}}\PY{p}{)}
              \PY{n}{plot2}\PY{o}{.}\PY{n}{set\PYZus{}ylabel}\PY{p}{(}\PY{l+s+s1}{\PYZsq{}}\PY{l+s+s1}{P\PYZhy{}value p(t)}\PY{l+s+s1}{\PYZsq{}}\PY{p}{)}
              \PY{n}{plot2}\PY{o}{.}\PY{n}{plot}\PY{p}{(}\PY{n}{time\PYZus{}array}\PY{p}{,} \PY{n}{pvalue}\PY{p}{,}\PY{l+s+s1}{\PYZsq{}}\PY{l+s+s1}{k}\PY{l+s+s1}{\PYZsq{}}\PY{p}{)}
              
              \PY{c+c1}{\PYZsh{} plot indicator vertical line on the graph for showing the maximum statistic time t\PYZus{}e}
              \PY{n}{plot1}\PY{o}{.}\PY{n}{plot}\PY{p}{(}\PY{p}{[}\PY{n}{t\PYZus{}e}\PY{o}{+}\PY{l+m+mi}{2} \PY{p}{,} \PY{n}{t\PYZus{}e}\PY{o}{+}\PY{l+m+mi}{2} \PY{p}{]}\PY{p}{,} \PY{p}{[}\PY{n+nb}{min}\PY{p}{(}\PY{n}{statistic}\PY{p}{)}\PY{p}{,}\PY{n+nb}{max}\PY{p}{(}\PY{n}{statistic}\PY{p}{)}\PY{p}{]}\PY{p}{,}\PY{l+s+s1}{\PYZsq{}}\PY{l+s+s1}{\PYZhy{}\PYZhy{}g}\PY{l+s+s1}{\PYZsq{}}\PY{p}{)}
              \PY{n+nb}{print}\PY{p}{(}\PY{l+s+s2}{\PYZdq{}}\PY{l+s+s2}{The maximum statistic is at this point: (time = }\PY{l+s+s2}{\PYZdq{}}\PY{p}{,}\PY{n}{time\PYZus{}array}\PY{p}{[}\PY{n}{t\PYZus{}e}\PY{o}{+}\PY{l+m+mi}{2}\PY{p}{]}\PY{p}{,}
                    \PY{l+s+s2}{\PYZdq{}}\PY{l+s+s2}{ ,Statistic S(t) =}\PY{l+s+s2}{\PYZdq{}}\PY{p}{,} \PY{n}{statistic}\PY{p}{[}\PY{n}{t\PYZus{}e}\PY{p}{]}\PY{p}{,}\PY{l+s+s2}{\PYZdq{}}\PY{l+s+s2}{ P\PYZhy{}value p(t) =}\PY{l+s+s2}{\PYZdq{}}\PY{p}{,} \PY{n}{pvalue}\PY{p}{[}\PY{n}{t\PYZus{}e}\PY{p}{]}\PY{p}{,}\PY{l+s+s2}{\PYZdq{}}\PY{l+s+s2}{)}\PY{l+s+s2}{\PYZdq{}}\PY{p}{)}
              \PY{n}{plot1}\PY{o}{.}\PY{n}{plot}\PY{p}{(}\PY{p}{[}\PY{n}{time\PYZus{}array}\PY{p}{[}\PY{n}{t\PYZus{}e}\PY{p}{]}\PY{p}{]}\PY{p}{,} \PY{p}{[}\PY{n}{statistic}\PY{p}{[}\PY{n}{t\PYZus{}e}\PY{p}{]}\PY{p}{]}\PY{p}{,} \PY{l+s+s1}{\PYZsq{}}\PY{l+s+s1}{go}\PY{l+s+s1}{\PYZsq{}}\PY{p}{)}
              \PY{n}{plot1}\PY{o}{.}\PY{n}{text}\PY{p}{(}\PY{n}{time\PYZus{}array}\PY{p}{[}\PY{n}{t\PYZus{}e}\PY{p}{]}\PY{o}{\PYZhy{}}\PY{l+m+mi}{24}\PY{p}{,} \PY{n}{statistic}\PY{p}{[}\PY{n}{t\PYZus{}e}\PY{p}{]}\PY{o}{+}\PY{l+m+mf}{1.1}\PY{p}{,} 
                         \PY{l+s+s2}{\PYZdq{}}\PY{l+s+s2}{ Statistic S(t) = }\PY{l+s+s2}{\PYZdq{}}\PY{o}{+}\PY{n+nb}{str}\PY{p}{(}\PY{n+nb}{round}\PY{p}{(}\PY{n}{statistic}\PY{p}{[}\PY{n}{t\PYZus{}star}\PY{p}{]}\PY{p}{,}\PY{l+m+mi}{4}\PY{p}{)}\PY{p}{)}\PY{p}{,} \PY{n}{color}\PY{o}{=}\PY{l+s+s1}{\PYZsq{}}\PY{l+s+s1}{green}\PY{l+s+s1}{\PYZsq{}}\PY{p}{)}
              
              \PY{n}{plot2}\PY{o}{.}\PY{n}{plot}\PY{p}{(}\PY{p}{[}\PY{n}{t\PYZus{}e}\PY{o}{+}\PY{l+m+mi}{2} \PY{p}{,} \PY{n}{t\PYZus{}e}\PY{o}{+}\PY{l+m+mi}{2} \PY{p}{]}\PY{p}{,} \PY{p}{[}\PY{n+nb}{min}\PY{p}{(}\PY{n}{pvalue}\PY{p}{)}\PY{p}{,}\PY{n+nb}{max}\PY{p}{(}\PY{n}{pvalue}\PY{p}{)}\PY{p}{]}\PY{p}{,}\PY{l+s+s1}{\PYZsq{}}\PY{l+s+s1}{\PYZhy{}\PYZhy{}g}\PY{l+s+s1}{\PYZsq{}}\PY{p}{)}
              \PY{n}{plot2}\PY{o}{.}\PY{n}{plot}\PY{p}{(}\PY{p}{[}\PY{n}{time\PYZus{}array}\PY{p}{[}\PY{n}{t\PYZus{}e}\PY{p}{]}\PY{p}{]}\PY{p}{,} \PY{p}{[}\PY{n}{pvalue}\PY{p}{[}\PY{n}{t\PYZus{}e}\PY{p}{]}\PY{p}{]}\PY{p}{,} \PY{l+s+s1}{\PYZsq{}}\PY{l+s+s1}{go}\PY{l+s+s1}{\PYZsq{}}\PY{p}{)}
              \PY{n}{plot2}\PY{o}{.}\PY{n}{text}\PY{p}{(}\PY{n}{time\PYZus{}array}\PY{p}{[}\PY{n}{t\PYZus{}e}\PY{p}{]}\PY{o}{\PYZhy{}}\PY{l+m+mi}{23}\PY{p}{,} \PY{n}{pvalue}\PY{p}{[}\PY{n}{t\PYZus{}e}\PY{p}{]}\PY{o}{+}\PY{l+m+mf}{0.1}\PY{p}{,} 
                         \PY{l+s+s2}{\PYZdq{}}\PY{l+s+s2}{ P\PYZhy{}value p(t) = }\PY{l+s+s2}{\PYZdq{}}\PY{o}{+}\PY{n+nb}{str}\PY{p}{(}\PY{n+nb}{round}\PY{p}{(}\PY{n}{pvalue}\PY{p}{[}\PY{n}{t\PYZus{}star}\PY{p}{]}\PY{p}{,}\PY{l+m+mi}{4}\PY{p}{)}\PY{p}{)}\PY{p}{,} \PY{n}{color}\PY{o}{=}\PY{l+s+s1}{\PYZsq{}}\PY{l+s+s1}{green}\PY{l+s+s1}{\PYZsq{}}\PY{p}{)}
              
              \PY{c+c1}{\PYZsh{} plot indicator vertical line on the graph for showing the real t\PYZus{}star}
              \PY{n}{plot1}\PY{o}{.}\PY{n}{plot}\PY{p}{(}\PY{p}{[}\PY{n}{t\PYZus{}star} \PY{p}{,} \PY{n}{t\PYZus{}star} \PY{p}{]}\PY{p}{,} \PY{p}{[}\PY{n+nb}{min}\PY{p}{(}\PY{n}{statistic}\PY{p}{)}\PY{p}{,}\PY{n+nb}{max}\PY{p}{(}\PY{n}{statistic}\PY{p}{)}\PY{p}{]}\PY{p}{,}\PY{l+s+s1}{\PYZsq{}}\PY{l+s+s1}{\PYZhy{}\PYZhy{}m}\PY{l+s+s1}{\PYZsq{}}\PY{p}{)}
              \PY{n}{plot1}\PY{o}{.}\PY{n}{plot}\PY{p}{(}\PY{p}{[}\PY{n}{time\PYZus{}array}\PY{p}{[}\PY{n}{t\PYZus{}star}\PY{o}{\PYZhy{}}\PY{l+m+mi}{2}\PY{p}{]}\PY{p}{]}\PY{p}{,} \PY{p}{[}\PY{n}{statistic}\PY{p}{[}\PY{n}{t\PYZus{}star}\PY{o}{\PYZhy{}}\PY{l+m+mi}{2}\PY{p}{]}\PY{p}{]}\PY{p}{,} \PY{l+s+s1}{\PYZsq{}}\PY{l+s+s1}{mo}\PY{l+s+s1}{\PYZsq{}}\PY{p}{)}
              \PY{n}{plot1}\PY{o}{.}\PY{n}{text}\PY{p}{(}\PY{n}{time\PYZus{}array}\PY{p}{[}\PY{n}{t\PYZus{}star}\PY{o}{\PYZhy{}}\PY{l+m+mi}{2}\PY{p}{]}\PY{o}{\PYZhy{}}\PY{l+m+mi}{24}\PY{p}{,} \PY{n}{statistic}\PY{p}{[}\PY{n}{t\PYZus{}star}\PY{o}{\PYZhy{}}\PY{l+m+mi}{2}\PY{p}{]}\PY{o}{+}\PY{l+m+mf}{0.8}\PY{p}{,} 
                         \PY{l+s+s2}{\PYZdq{}}\PY{l+s+s2}{ Statistic S(t) = }\PY{l+s+s2}{\PYZdq{}}\PY{o}{+}\PY{n+nb}{str}\PY{p}{(}\PY{n+nb}{round}\PY{p}{(}\PY{n}{statistic}\PY{p}{[}\PY{n}{t\PYZus{}star}\PY{p}{]}\PY{p}{,}\PY{l+m+mi}{4}\PY{p}{)}\PY{p}{)}\PY{p}{,} \PY{n}{color}\PY{o}{=}\PY{l+s+s1}{\PYZsq{}}\PY{l+s+s1}{purple}\PY{l+s+s1}{\PYZsq{}}\PY{p}{)}
              
              \PY{n}{plot2}\PY{o}{.}\PY{n}{plot}\PY{p}{(}\PY{p}{[}\PY{n}{t\PYZus{}star} \PY{p}{,} \PY{n}{t\PYZus{}star} \PY{p}{]}\PY{p}{,} \PY{p}{[}\PY{n+nb}{min}\PY{p}{(}\PY{n}{pvalue}\PY{p}{)}\PY{p}{,}\PY{n+nb}{max}\PY{p}{(}\PY{n}{pvalue}\PY{p}{)}\PY{p}{]}\PY{p}{,}\PY{l+s+s1}{\PYZsq{}}\PY{l+s+s1}{\PYZhy{}\PYZhy{}m}\PY{l+s+s1}{\PYZsq{}}\PY{p}{)}
              \PY{n}{plot2}\PY{o}{.}\PY{n}{plot}\PY{p}{(}\PY{p}{[}\PY{n}{time\PYZus{}array}\PY{p}{[}\PY{n}{t\PYZus{}star}\PY{o}{\PYZhy{}}\PY{l+m+mi}{2}\PY{p}{]}\PY{p}{]}\PY{p}{,} \PY{p}{[}\PY{n}{pvalue}\PY{p}{[}\PY{n}{t\PYZus{}star}\PY{o}{\PYZhy{}}\PY{l+m+mi}{2}\PY{p}{]}\PY{p}{]}\PY{p}{,} \PY{l+s+s1}{\PYZsq{}}\PY{l+s+s1}{mo}\PY{l+s+s1}{\PYZsq{}}\PY{p}{)}
              \PY{n}{plot2}\PY{o}{.}\PY{n}{text}\PY{p}{(}\PY{n}{time\PYZus{}array}\PY{p}{[}\PY{n}{t\PYZus{}star}\PY{o}{\PYZhy{}}\PY{l+m+mi}{2}\PY{p}{]}\PY{o}{\PYZhy{}}\PY{l+m+mi}{23}\PY{p}{,} \PY{n}{pvalue}\PY{p}{[}\PY{n}{t\PYZus{}star}\PY{o}{\PYZhy{}}\PY{l+m+mi}{2}\PY{p}{]}\PY{o}{+}\PY{l+m+mf}{0.3}\PY{p}{,} 
                         \PY{l+s+s2}{\PYZdq{}}\PY{l+s+s2}{ P\PYZhy{}value p(t) = }\PY{l+s+s2}{\PYZdq{}}\PY{o}{+}\PY{n+nb}{str}\PY{p}{(}\PY{n+nb}{round}\PY{p}{(}\PY{n}{pvalue}\PY{p}{[}\PY{n}{t\PYZus{}star}\PY{p}{]}\PY{p}{,}\PY{l+m+mi}{4}\PY{p}{)}\PY{p}{)}\PY{p}{,} \PY{n}{color}\PY{o}{=}\PY{l+s+s1}{\PYZsq{}}\PY{l+s+s1}{purple}\PY{l+s+s1}{\PYZsq{}}\PY{p}{)}
              
          \PY{n}{t\PYZus{}test\PYZus{}plots}\PY{p}{(}\PY{n}{samples}\PY{p}{,} \PY{n}{alpha}\PY{o}{=}\PY{l+m+mf}{0.05}\PY{p}{)}
\end{Verbatim}


    \begin{Verbatim}[commandchars=\\\{\}]
The computed change point index (t\_e) =  226
The maximum statistic is at this point: (time =  228  ,Statistic S(t) = -3.17096343151  P-value p(t) = 0.00167761737457 )

    \end{Verbatim}

    \begin{center}
    \adjustimage{max size={0.9\linewidth}{0.9\paperheight}}{output_5_1.png}
    \end{center}
    { \hspace*{\fill} \\}
    
    \hypertarget{some-explanation-about-the-python-implementation}{%
\subsubsection{Some explanation about the python
Implementation}\label{some-explanation-about-the-python-implementation}}

For generating Gaussian random number I used
\(\texttt{np.random.normal()}\) function but at first I used
\(\texttt{stats.norm.rvs()}\) which the corresponding line of code
commented at the code block. For testing them you only need to uncomment
them and comment the one which is running now.

As you can see from the above figure, the maximum absolute value for the
statistic \(S(t)\) is \(2.0446\) which correspond to the \(t_e=226\),
the green vertical line on the graphs. The calculated p-value for this
index of the time is \(p(t) = 0.0418\). By comparing this value to
\(alpha=0.05\) this point is chosen as change point in our observation
sequence.

However, the actual and real time of change, \(t^\ast\), is now exactly
what we caculated at this change point method. This point presented by
purple vertical line on both graphs.

To answer this question that ``Does the CPM detected a change?'' my
answer is yes. It was not as accurate as it must be to identify change
point but still acceptable the real value \(t^\ast = 200\) and
\(t_e=226\).

The associated confidence level of this point can be computed using this
formula \(\alpha = 1 - p\) which \(\alpha\) is significance level and
\(p\) is the confidence level. Thus the confidence level \(p\) for
\(t_e = 226\) is equal to: \[p = 1 - 0.0418 = 0.9582\]

This confidence level is slightly greater that its limit \(0.95\) which
means that we can reject the null hypothesis and consider this point as
change point.

Finally, I want to note that the function \(\texttt{stats.ttest_ind}\)
that I used to compute the statistical t-test is calculating the
two-sided (two-tailed) test. And if I you want to compute the one-tailed
test the only thing you need to do is just comparing the
\(\texttt{pvalue/2}\) with \(\alph\) instead.

    \hypertarget{references}{%
\subsection{References:}\label{references}}

Python for Data Analysis Part 24: Hypothesis Testing and the T-Test.
http://hamelg.blogspot.com/2015/11/python-for-data-analysis-part-24.html

Python Scipy library documentation
https://docs.scipy.org/doc/scipy/reference/generated/scipy.stats.ttest\_ind.html

How to perform two-sample one-tailed t-test with numpy/scipy?
https://stackoverflow.com/questions/15984221/how-to-perform-two-sample-one-tailed-t-test-with-numpy-scipy

Understanding Hypothesis Tests: Significance Levels (Alpha) and P values
in Statistic
http://blog.minitab.com/blog/adventures-in-statistics-2/understanding-hypothesis-tests-significance-levels-alpha-and-p-values-in-statistics

What are the differences between one-tailed and two-tailed tests?
https://stats.idre.ucla.edu/other/mult-pkg/faq/general/faq-what-are-the-differences-between-one-tailed-and-two-tailed-tests/


    % Add a bibliography block to the postdoc
    
    
    
    \end{document}
